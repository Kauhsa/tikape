\documentclass{article}

\usepackage[defaultmono]{droidmono}
\usepackage[T1]{fontenc}
\makeatletter
\input droid/t1fdm.fd
\makeatother

\usepackage[utf8]{inputenc}
\usepackage[finnish]{babel}
\usepackage{hyperref}
\usepackage{graphicx}
\usepackage{listings}
\usepackage{color}

\setlength{\parindent}{0.0in}
\setlength{\parskip}{0.1in}

\let\stdsection\section
\renewcommand\section{\newpage\stdsection}

\definecolor{dkgreen}{rgb}{0,0.6,0}
\definecolor{gray}{rgb}{0.5,0.5,0.5}
\definecolor{mauve}{rgb}{0.58,0,0.82}

\lstset{ %
  backgroundcolor=\color{white},  % choose the background color; you must add \usepackage{color} or \usepackage{xcolor}
  basicstyle=\ttfamily\scriptsize,       % the size of the fonts that are used for the code
  breaklines=true,                % sets automatic line breaking
  commentstyle=\color{dkgreen},   % comment style
  deletekeywords={...},           % if you want to delete keywords from the given language
  escapeinside={\%*}{*)},         % if you want to add LaTeX within your code
  %frame=single,                   % adds a frame around the code
  keywordstyle=\color{blue},      % keyword style
  numberstyle=\tiny\color{gray},  % the style that is used for the line-numbers
  rulecolor=\color{black},        % if not set, the frame-color may be changed on line-breaks within not-black text (e.g. comments (green here))
  stringstyle=\color{mauve},      % string literal style
  showspaces=false,
  showstringspaces=false,
  literate={ö}{{\"o}}1
           {ä}{{\"a}}1,
  language=SQL,
  morekeywords={SERIAL, REFERENCES}
}

\title{Tietokantojen perusteet 2013, Periodi III \\ Kirjastotietokanta}
\author{Mika Viinamäki \\ Paavo Rohamo \\ John Lång \\ Eero Antila \\ Juha Koiranen}

\begin{document}

\pagenumbering{gobble}
\begin{titlepage}
\maketitle
\end{titlepage}

\pagenumbering{arabic}
\setcounter{page}{2}

\section{Tietosisältökartoitus}

Tietokohteet \textbf{lihavoituna}, attribuutit \textit{kursivoituna}.

Kirjaston kokoelmissa on noin 120000 eri \textbf{kohdetta}. Suurin osa kohteista on \textit{lainattavia}, mutta osa kuuluu käsikirjastoon. Erilaisia \textbf{nimikkeitä} on noin 80000. Kullakin nimikkeellä on \textit{tunnus}, \textit{nimi} ja \textit{tyyppi}. Lisäksi nimikkeeseen voi liittyä runsaasti erilaista \textbf{kuvailutietoa}, esimerkiksi tekijä, kustantaja, sivulukumäärä, jne. Kullakin \textbf{kuvailutietotyypillä} on yksikäsitteinen tunnus. Esimerkiksi kustantajatiedon \textit{tyyppitunnus} on PUBL. \textbf{Nimiketyyppi}kohtaisesti (tyyppejä esimerkiksi kirja ja elokuva) on määritelty, mitä kuvailutietoja kyseisen tyypin nimikkeisiin liittyy sekä mitkä niistä ovat \textit{pakollisia} ja mitkä valinnaisia. Uusia kuvailutietotyyppejä pitää pystyä lisäämään ja kytkemään nimikkeisiin muuttamatta tietokannan rakennetta. Kuvailutietoja käytetään lähinnä nimiketietojen haussa.

Kohdekohtaiset tiedot eivät riipu kohteen tyypistä. Kullakin kohteella on yksikäsitteinen \textit{tunnusnumero}. Lisäksi kohteesta säilytetään \textbf{sijaintitietoa} (\textit{osasto}, \textit{hyllykkö}, \textit{hylly}) ja tietoa \textit{hankinta-ajankohdasta} ja \textit{hankintahinnasta}.

Kirjaston asiakaskunta muodostuu noin 20000 \textbf{asiakkaasta}. Asiakkaasta on taltioitu \emph{normaalit asiakastiedot}. Asiakkaalla on yksikäsitteinen \emph{asiakasnumero}. Asiakkaat voivat \textbf{lainata} kohteita. He voivat myös \textbf{tilata} nimikkeitä. Palautuksen yhteydessä järjestelmä ilmoittaa ensimmäiselle jonottavalle asiakkaalle kohteen saapuomisesta. Lainasta tallennetaan \emph{lainausajankohta} ja \emph{palautuspäivä}. Lainaustietoihin voi liittyä myös \emph{karhuamismerkintöjä}. Kun laina palautetaan, lainaustietoa ei poisteta, vaan se jää kantaan historiatiedoksi. 

\section{Käsitekaavio}

\includegraphics[width=\linewidth]{../kaaviot/kasitekaavio.pdf}

Nimike kuvaa jotain kirjaston valikoimissa olevaa teosta, ja kohde kuvaa yhtä fyysistä kopiota jostain tällaisesta nimikkeestä.

Nimiketyyppi kuvaa jotain muotoa, jota nimike edustaa - esimerkiksi kirja tai elokuva. Jokaiseen nimiketyyppiin liittyy tietyt kuvailutietotyypit, jotka voivat olla pakollisia tai vapaaehtoisia. Kuvailutieto tarkoittaa jotain nimikkeeseen liittyvää tietoa, kuten esimerkiksi nimikkeen kustantajaa tai ohjaajaa. Kuvailutietotyyppi kuvaa tämän tiedon tyyppiä.

Tilaus kuvaa varausta johonkin nimikkeeseen.

\section{Tietokantakuvaus}

\subsection{Tietokantakaavio}

\includegraphics[width=\linewidth]{../kaaviot/tietokantakaavio.pdf}

\subsection{SQL-lauseet}

Kaikki tämän dokumentin SQL-lauseet olettavat käytössä olevan PostgreSQL-tietokantaohjelmiston tuoreehko versio (testattu versiolla 9.2.3).

\lstinputlisting{../sql/create_tables.sql}

\section{Riippuvuusanalyysi}

\textbf{Asiakas:} Taulussa on funktionaalisia riippuvuuksia esimerkiksi postinumerolla ja postitoimipaikalla - jälkimmäinen riippuu suoraan ensimmäisestä. Taulu ei siis täytä kolmatta normaalimuotoa.

\textbf{Sijainti, Nimiketyyppi, Nimike, Kuvailutietotyyppi, Kuuluvat\_\-Kuvailutietotyypit, Kuvailutieto, Kohde, Tilaus:} Attribuuteilla ei funktionaalisia riippuvuuksia kuin pääavaimiin.

\textbf{Laina:} Palautettava-attribuutin voi päätellä lainattu-attribuutin ja Kohde-taulun laina-ajan perusteella - olettaen että kohteen laina-aika ei koskaan muutu. Kohteen laina-aika voi kuitenkin muuttua, jolloin kyseessä ei ole funktionaalinen riippuvuus.

\section{Esimerkkidata}

Esimerkkidata olettaa, että tietokanta on tyhjä ennen näiden SQL-lauseiden suorittamista --- koska \verb+SERIAL+-tyyppisten attribuuttien arvoja ei \verb+INSERT+-lauseissa ole annettu, relaatiot voivat mennä jotenkin muuten kuin on tarkoitettu mikäli tietokannassa on jo rivejä.

\lstinputlisting{../sql/initial_data.sql}

\section{Käyttötapauksia}

\subsection{Kaikkien kohteiden listaaminen}

Kaikki tietokannassa olevat kohteet, niiden nimet ja tyypit selkokielisenä sekä sijaintitiedot saa seuraavalla kyselyllä:

\begin{lstlisting}
SELECT NIMIKE.Nimi,
    NIMIKETYYPPI.Nimi,
    SIJAINTI.Osasto,
    SIJAINTI.Hyllykko,
    SIJAINTI.Hylly
FROM KOHDE
INNER JOIN NIMIKE
    ON KOHDE.NimikeID = NIMIKE.NimikeID
INNER JOIN NIMIKETYYPPI
    ON NIMIKE.NimikeTyyppiID = NIMIKETYYPPI.NimikeTyyppiID
INNER JOIN SIJAINTI
    ON KOHDE.SijaintiID = SIJAINTI.SijaintiID;
\end{lstlisting}

\subsection{Nimikkeen kuvailutietojen listaaminen}

Tietyn nimikkeen (tässä tapauksessa NimikeID = 1) kuvailutiedot kuvailutietotyyppien selkokielisten nimien kera:

\begin{lstlisting}
SELECT KUVAILUTIETOTYYPPI.Nimi,
    KUVAILUTIETO.Tieto 
FROM KUVAILUTIETO
INNER JOIN KUVAILUTIETOTYYPPI
    ON KUVAILUTIETO.KuvailutietoTyyppiID = KUVAILUTIETOTYYPPI.KuvailutietoTyyppiID
WHERE
    KUVAILUTIETO.NimikeID = 1
ORDER BY Nimi;
\end{lstlisting}

\subsection{Nimikkeiden hakeminen kuvailutietojen perusteella}

Esimerkkinä kaikki nimikkeet, joiden kustantaja on "Pekka":

\begin{lstlisting}
SELECT DISTINCT NIMIKE.NimikeID, NIMIKE.Nimi
FROM NIMIKE
INNER JOIN KUVAILUTIETO
    ON NIMIKE.NimikeID = KUVAILUTIETO.NimikeID
WHERE
    KUVAILUTIETO.KuvailutietoTyyppiID = 'PUBL'
AND
    KUVAILUTIETO.Tieto = 'Pekka'
\end{lstlisting}

\subsection{Kohteen lainaaminen}

Lainaustapahtuman yhteydessä tiedämme kohteen \texttt{KohdeID}:n - se saadaan esimerkiksi kohteen takana olevasta viivakoodista. Esimerkeissä \texttt{KohdeID} on 1. Lisäksi asiakkaan \texttt{AsiakasID} on tiedossa - se saadaan vaikkapa asiakkaan kirjastokortista. Esimerkeissä \texttt{AsiakasID} on niin ikään 1.

Ihan ensin tarkistetaan, onko kohde ylipäätään paikalla ja onko se lainattavissa --- jos ei ole, lainausta ei voida tehdä.

\begin{lstlisting}
SELECT EXISTS (
    FROM KOHDE
    WHERE
        KOHDE.KohdeID = 1
    AND
        KOHDE.LainaAika > 0
    AND
        NOT EXISTS (
            SELECT 1
            FROM Laina
            WHERE
                KohdeID = KOHDE.KohdeID
            AND
                Palautettu IS NULL
        );
)
\end{lstlisting}

Tästä tarkistuksesta selvittyämme selvitämme kohteen \texttt{NimikeID} --- sitä tarvitaan useaan otteeseen:

\begin{lstlisting}
SELECT NimikeID
FROM KOHDE
WHERE KohdeID = 1
LIMIT 1;
\end{lstlisting}

Oletetaan, että kohteen \texttt{NimikeID}:ksi saatiin 2.

Nyt tarkistetaan, onko nimikkeestä enemmän kohteita paikalla kuin avoimia tilauksia - jos ei ole, lainausta ei voida tehdä. Tilausten määrä tälle nimikkeelle saadaan seuraavan kyselyn avulla:

\begin{lstlisting}
SELECT count(*)
FROM TILAUS
WHERE TILAUS.NimikeID = 2;
\end{lstlisting}

Lainattavissa ja paikalla olevien kyseisen nimikkeen kohteiden määrä taas seuraavasti:

\begin{lstlisting}
SELECT count(*) 
FROM KOHDE
WHERE
    KOHDE.NimikeID = 2
AND
    KOHDE.LainaAika > 0
AND
    NOT EXISTS (
        SELECT 1
        FROM Laina
        WHERE
            KohdeID = KOHDE.KohdeID
        AND
            Palautettu IS NULL
    );
\end{lstlisting}

Jos tilauksia on enemmän tai yhtä monta kuin saatavilla olevia kohteita, varausta ei voida tehdä - ellei asiakas ole niin monen varauksen joukossa kuin saatavilla olevia kohteita on. Oletetaan, että paikalla olevia kohteita halutulle nimikkeelle on 5:

\begin{lstlisting}
SELECT EXISTS (
    SELECT 1
    FROM (
        SELECT NimikeID, AsiakasID
        FROM TILAUS
        ORDER BY Paivamaara DESC
        LIMIT 5
    ) AS T
    WHERE
        NimikeID = 2
    AND
        AsiakasID = 1
);
\end{lstlisting}

Jos kysely palauttaa \texttt{TRUE}, voidaan jatkaa vaikka saatavilla olevia kohteita ei olisikaan enempää kuin varauksia kyseiselle nimikkeelle. Nyt kun kaikki on kunnossa, voidaan kirjata uusi lainaus:

\begin{lstlisting}
INSERT INTO LAINA (AsiakasID, KohdeID, Lainattu, Palautettava) VALUES (
    1,
    1,
    CURRENT_DATE,
    CURRENT_DATE + (SELECT LainaAika FROM KOHDE WHERE KohdeID = 1)
);
\end{lstlisting}

Lopuksi voidaan poistaa kyseisen asiakkaat kyseiselle nimikkeelle tekemät varaukset, koska asiakas on saanut kohteensa:

\begin{lstlisting}
DELETE FROM TILAUS WHERE AsiakasID = 1 AND NimikeID = 2;
\end{lstlisting}

\section{Ratkaisun ongelmat}
Tietokannassa ei välttämättä ole kaikkia haluttuja ominaisuuksia --- tietokantaan ei esimerkiksi pysty tallentamaan tietoa esimerkiksi asiakkaiden myöhästyneistä palautuksista aiheutuneista veloista tai niiden maksuista.

Lisäksi koska kaikki kuvailutieto tallennetaan \verb+VARCHAR+-tyyppiseen attribuuttiin, voi siitä aiheutua jotain ongelmia --- esimerkiksi hakuihin joissa halutaan nimikkeitä jotka on julkaistu tietyllä vuosivälillä tulee ylimääräinen mutka matkaan.

Toinen ongelma on esimerkiksi tilauksen ajankohdan tallentaminen vain päivän tarkkuudella - sama päivänä tehtyjä tilauksia ei saa oikeaan järjestykseen, ellei sitten luota siihen että \verb+TilausID+ on aina kasvava.


\end{document}
